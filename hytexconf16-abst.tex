\documentclass[a4paper,twocolumn]{jsarticle}
\usepackage[dvipdfm,margin=20truemm]{geometry}
\usepackage{hytexconf16} % personal settings
\makeatletter
% set box before loading newpxtext/newpxmath (ensure OT1 encoding)
\newbox\TEMPAA
\setbox\TEMPAA=\hbox{\tate\adjustbaseline{\AA}}
\edef\TEMPENC{\f@encoding}
% all done
\makeatother
\usepackage{plext}
\usepackage{newpxtext,newpxmath}
\pagestyle{empty}

\AtBeginDvi{\special{pdf:mapfile hiraginopron.map}}

\title{\TeX\ Live 2016の新しい\pLaTeX}
\author{山下 弘展(Hironobu Yamashita)}
\date{}
\begin{document}
\maketitle

\pLaTeX は日本語組版のデファクトスタンダードとなっている.
しかし,幾つかの組版上の問題点や\LaTeX の修正との不整合も指摘されていた.
本講演では,\TeX\ Live 2016以降に収録されている,
日本語\TeX 開発コミュニティによる新しい\pLaTeX の現状を整理する.

\section{コミュニティ版\pLaTeX の概要}
株式会社アスキー
(現アスキー・メディアワークス\footnote{\texttt{http://asciimw.jp}})に
よって開発された\pLaTeX は,\pTeX とともに2010年に\TeX\ Liveへと
取り込まれた.以来,既定エンジンの\pTeX から\epTeX への変更,
\epTeX の仕様変更,さらには\pLaTeX のベースである\LaTeX の修正など,
\pLaTeX の周辺は大きく変化を遂げた.

こうした流れのなかで,\pLaTeX は2006年のアスキーによる最後のリリース
から10年近くが経過し,こうした変化による影響が無視できなくなってきた.
そこで,\pTeX/\epTeX との不整合を解消することを主目的として,
2016年1月に日本語\TeX 開発コミュニティ(Japanese \TeX\ Development
Community\footnote{\texttt{https://texjp.org}})が\pLaTeX をforkした.
これがコミュニティ版
\pLaTeX\footnote{\texttt{https://github.com/texjporg/platex}}で
あり,\upLaTeX\footnote{\texttt{https://github.com/texjporg/uplatex}}も
これと同期する形で開発されている.

\section{\TeX\ Live 2016での\pLaTeX の変更点}
主なものを箇条書きで列挙する.詳細は\pLaTeX とともに配布されている
\file{plnewsc*.tex}を参照されたい.
\begin{itemize}
  \item \pTeX の修正への追随:
        脚注の合印前後,|tabular|環境前後などの
        不自然なアキを削除
  \item \LaTeX への追随:
        |\eminnershape|による強調書体変更機能への適合,
        \file{latexrelease}パッケージへの対応
  \item 組版処理の改善:合印直後の改行が抑制される問題を解消,
        合印直前のベタ組を実現
  \item 8bit font encodingへの対応:文字コード128--255の欧文文字前後の
        |\xkanjiskip|アキを設定
  \item 縦組の不具合修正,ascmacパッケージの修正
\end{itemize}

\section{今後の\pLaTeX の開発課題}
現行のコミュニティ版\pLaTeX も,まだ従来から指摘されているいくつかの
不都合が残っている.特に縦組で問題になる現象が多く,原因も判明して
いるが,修正により生じる副作用に注意しつつ検討する余地がある.

以下に2点列挙する:
\begin{itemize}
\item ベースライン補正量|\{y,t}baselineshift|がゼロでない場合に,
      アクセント合成文字が乱れる
\item 縦組で\file{amsmath}の|align|環境を使うと,
      数式の位置や数式番号がずれる
\end{itemize}
入力例と現状の縦組による出力例を以下に示す:
  \begin{quote}\begin{verbatim}
OT1 encodingの合成文字\AA が変\par
align環境\texttt{\&}が1つ %% 少し上へ
\begin{align}
  a_1 &= b_1+c_1\\
  a_2 &= b_2+c_2-d_2
\end{align}
align環境\texttt{\&}なし %% 端に付く
\begin{align}
  a_1=b_1+c_1
\end{align}
比較用のequation環境
\begin{equation}
  a_1=b_1+c_1
\end{equation}
  \end{verbatim}\end{quote}
  \begin{center}
    \setlength{\fboxsep}{10pt}
      \fbox{\begin{minipage}<t>{15zw}
        \TEMPENC\ encodingの合成文字\box\TEMPAA が変\par
        align環境\texttt{\&}が1つ %% 少し上へ
        \begin{align}
          a_1 &= b_1+c_1\\
          a_2 &= b_2+c_2-d_2
        \end{align}
        align環境\texttt{\&}なし %% 端に付く
        \begin{align}
          a_1=b_1+c_1
        \end{align}
        比較用のequation環境
        \begin{equation}
          a_1=b_1+c_1
        \end{equation}
      \end{minipage}}
  \end{center}
\end{document}
